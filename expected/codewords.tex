\documentclass[a5paper,onecolumn]{book}

\usepackage[utf8]{inputenc}
\usepackage[T1]{fontenc}
\usepackage[hidelinks]{hyperref}
\usepackage{graphicx}

\usepackage[top=3.3cm, bottom=3.3cm, left=2cm, right=2cm]{geometry}
\newif\ifpdf
\ifx\pdfoutput\undefined
  \pdffalse
\else
  \ifnum\pdfoutput=1
    \pdftrue
  \else
    \pdffalse
  \fi
\fi

\title{Gamebook}
\author{}
\date{}

\newcounter{sectionnr}

\begin{document}

\maketitle

\clearpage

\thispagestyle{empty}

\pagestyle{empty}


Turn to 1 to begin.
\phantomsection
\refstepcounter{sectionnr}
\label{section1}
\subsection*{\begin{center} \textbf{1} \end{center}}

 \noindent
 Demonstrating how Codewords (AKA sightings) can be used. Go to \textbf{\autoref{section2}}. 
\vspace{1em}
\phantomsection
\refstepcounter{sectionnr}
\label{section2}
\subsection*{\begin{center} \textbf{2} \end{center}}

 \noindent
 Got codeword \textbf{warrior}. Simple enough to set a codeword. Turn to \textbf{\autoref{section9}}. 
\vspace{1em}
\phantomsection
\refstepcounter{sectionnr}
\label{section3}
\subsection*{\begin{center} \textbf{3} \end{center}}

 \noindent
 That was easy.    
\vspace{1em}
\phantomsection
\refstepcounter{sectionnr}
\label{section4}
\subsection*{\begin{center} \textbf{4} \end{center}}

 \noindent
 OK, if you have the codeword \textit{warrior}, you may turn to \textbf{\autoref{section3}} otherwise you may go back to \textbf{\autoref{section2}}. Although we both know you have that codeword. If you have the codeword \textit{fun} you may turn to \textbf{\autoref{section5}}, without it you can go to \textbf{\autoref{section8}}. 
\vspace{1em}
\phantomsection
\refstepcounter{sectionnr}
\label{section5}
\subsection*{\begin{center} \textbf{5} \end{center}}

 \noindent
 Cheater! There is no way you can have codeword fun. 
\vspace{1em}
\phantomsection
\refstepcounter{sectionnr}
\label{section6}
\subsection*{\begin{center} \textbf{6} \end{center}}

 \noindent
 If you have the codeword \textit{fun} turn to \textbf{\autoref{section5}}. Otherwise you can go to the end at \textbf{\autoref{section3}} or to the xortest at \textbf{\autoref{section4}}. 
\vspace{1em}
\phantomsection
\refstepcounter{sectionnr}
\label{section7}
\subsection*{\begin{center} \textbf{7} \end{center}}

 \noindent
 If you have codeword \textit{warrior} turn to \textbf{\autoref{section6}}. Otherwise you can go to the end at \textbf{\autoref{section3}} or back to \textbf{\autoref{section1}}. 
\vspace{1em}
\phantomsection
\refstepcounter{sectionnr}
\label{section8}
\subsection*{\begin{center} \textbf{8} \end{center}}

 \noindent
 This is just to demonstrate choices allowed when not having a codeword. Now go on to \textbf{\autoref{section7}} (autotest) or \textbf{\autoref{section4}} (xor test). 
\vspace{1em}
\phantomsection
\refstepcounter{sectionnr}
\label{section9}
\subsection*{\begin{center} \textbf{9} \end{center}}

 \noindent
 If you have the codeword \textit{warrior} you may turn to \textbf{\autoref{section3}}. If you do not have the codeword \textit{fun}, you may turn to \textbf{\autoref{section8}}. If you have the codeword \textit{fun}, you may instead turn to \textbf{\autoref{section5}}. Otherwise see \textbf{\autoref{section4}}. 
\vspace{1em}
\end{document}
